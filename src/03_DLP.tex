\chapter{Problème du Logarithme Discret}
\todo{retrouver ma fiche sur DLP}
Survey de février 2014 d'Antoine Joux sur le DLP \href{https://www1.lip6.fr/~pierrot/papers/DlogSurvey.pdf}{Survey}.

\todo{Introduction du problème (DLP, DDLP, CDLP + EC), définir les notations, algorithme générique (black-box groups), théorème de Shoup, objectif en crypto, recherche exhaustive \ldots}

D'un côté, on a les algorithmes génériques. Le théorème de Shoup fournit une borne inférieure sur la complexité d'un algorithme générique. On connaît des algorithmes optimaux de résolution du problème du logarithme discret dans un cadre générique. 

D'un autre côté, on a les algorithmes non génériques qui vise à tirer partie des spécificités du groupe afin d'obtenir des algorithmes de complexité sous-exponentielles c-à-d meilleurs que les algorithmes génériques. 

Le groupe des points d'une courbe elliptique est optimal pour le DLP. Excepté pour quelques familles de courbes, aucun algorithme non génériques existe.

\section{Algorithmes génériques}

\subsection{Algorithme de Silver-Pohlig-Helman}
\begin{itemize}[label=$\rightarrow$]
    \item Cet algorithme a des répercussions importantes.
    \item En pratique, si l'ordre du point de base est quelconque alors on utilise cet algorithme puis pour chaque sous-problème, une des méthodes suivantes est utilisées. 
\end{itemize}


\subsection{Baby Step Giant Step (BSGS)}
Aussi appelé algorithme de Shanks.

\subsection{Algorithme Rho de Pollard}
Paradoxe des anniversaires : combien doit-on réunir de personnes afin que la probabilité que deux d'entre eux aient la même date d'anniversaire soit supérieur à $1/2$ ? $23$ suffit ! Il ne s'agit pas en réalité d'un paradoxe mais seulement d'un fait contre-intuitif. 

Dans un groupe de $n$ personnes quelle est la probabilité que deux d'entre elles fêtent leur anniversaire le même jour ? On se donne $\mathcal{U}$ un ensemble de cardinal $n$. On effectue $k$ tirages avec remise. Les boules sont tirées de manières uniforme.
\begin{itemize}
    \item Quelle est la probabilité d'obtenir une collision ? ie : calculer $\mathcal{P}(n, k)$, tirer deux fois la même boule. Ou de manière analogue, par complémentarité, quelle est la probabilité que les boules tirées soient toutes distinctes ? ie : que l'on obtienne aucune collision, calculer  $1 - \mathcal{P}(n, k)$. Dans ce cas, $k < n$ sinon cette probabilité est nulle. 
    \item Fixons $n$ la taille de notre urne. On se donne un seuil $\alpha \in [0, 1]$. Combien doit-on effectuer de tirages afin d'obtenir $\mathcal{P}(n, k) \geq \alpha$ ? ie : obtenir une collision avec une probabilité supérieur ou égal à $\alpha$.
    \item Théorème : Si des éléments sont tirés de manière aléatoire d'un ensemble $\mathcal{U}$ alors le nombre de tirage attendu avant d'obtenir une collision est $\sqrt{\pi n / 2}$ où $n = \# U$.
\end{itemize}

Le problème des anniversaires se reformule en terme de collision ce qui aura son importance lors de l'étude des fonctions de hachages.

Si une fonction de hachage a une sortie de $N$ bits, alors 

\href{http://www.lifl.fr/~wegrzyno/portail/PAC/Doc/Cours/paradoxeAnniversaire.pdf}{Lien}


\subsection{Algorithme du kangourou de Pollard}
Pollard’s Kangaroo method is based on running two independent random walks on a cyclic group G, one starting at a known state (the "tame kangaroo") and the other starting at the unknown but nearby value of the discrete logarithm x (the "wild kangaroo"), and terminates after the first intersection of the walks. As such, in order to analyze the algorithm it suffices to develop probabilistic tools for examining the expected time until independent random walks on a cyclic group intersect, in terms of some measure of the initial distance between the walks.

Cette méthode est appelée soit la méthode lambda de Pollard ou la méthode du kangourou apprivoisé et sauvage.


\subsection{Logarithme discret en groupe}
\todo{Mettre la footnote dans le titre.}

\footnote{Batch discrete logarithm.}
Cf. Safecurves et article de générations de courbes à la volée.


\section{Algorithmes non génériques}
\subsection{Calcul d'indice}
\todo{Rappeler le principe général de l'algorithme dans \GF{p}. Donner les derniers résultats dans ce domaine cf Antoine Joux et ses doctorants. Mentionner l'existence d'un algorithme similaire pour les courbes hyperelliptiques de genre supérieur ou égal à 3. Expliquer brièvement pourquoi il n'existe pas d'algorithme similaire aux courbes hyperelliptiques de genre 1 et 2.}


\subsection{Xedni calculus}
Algorithme de Silverman en 98. Le nom Xedni vient de indeX à l'envers. Allez voir \cite{koblitz2011elliptic}.

\todo{Donner seulement les conséquences de cet algorithme, cf article de Koblitz sur l'histoire d'ECC.}