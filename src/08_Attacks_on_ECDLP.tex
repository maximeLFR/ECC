\chapter{ECDLP security}
\section{Attaques génériques}
Parler du rho de pollard et de l'algorithme de Pohlig Helman. Et des conséquences sur l'ordre du groupe des points d'une courbe elliptique.

\section{Transfert du logarithme discret}
Une stratégie pour attaquer le DLP sur courbes elliptiques est de le transférer, en exhibant un isomorphisme, dans un groupe où l'on est capable de résoudre le DLP plus facilement. Pour être réalisable, il est nécessaire que l'application soit facilement calculable et que le groupe ne soit pas trop grand. Nous allons voir que les couplages nous permettent de construire de tels isomorphismes pour des familles de courbes bien spécifique. Ce sont d'ailleurs les premières applications des couplages en cryptographie.


\subsection{Transfert additif}
Cette attaque, présentée en 98/99 indépendamment par Smart \cite{smart1999discrete}, Semaev et Satoh-Araki \cite{satoh1998fermat}, réussit à expliciter, pour une certaine famille de courbes, un isomorphisme facilement calculable entre $(E(\GF{q}), +)$ et $(\GF{q}, +)$. 

\begin{definition}
Soit $E$ une courbe elliptique définie sur \GF{q}. La courbe elliptique $E$ est dite \emph{anomale sur \GF{q}} si et seulement si $\# E(\GF{q}) = q$. C'est à dire si et seulement si la trace vaut un.
\end{definition}

Il faut distinguer le corps de base qui le plus petit corps contenant les coefficients régissant l'équation de la courbe du corps des points rationnels qui est le corps où sont définis les coordonnées des points de la courbe auxquelles on s'intéresse. En cryptographie, lorsqu'on utilise des courbes sur corps premier, les deux sont confondues. L'anomalité d'une courbe ainsi que la trace sont des notions dépendantes du corps des points rationnels. On devrait ainsi noter en toute rigueur $t_{\GF{q}}$, lorsque rien n'est précisé c'est le corps de base qui est sous-entendu.

Les courbes sujettes à cette attaque sont les courbes anomales sur le corps des points rationnels. Elles peuvent donc être anomale sur le corps de base sans pour autant l'être sur le corps des points rationnels et ainsi ne pas être sujette à l'attaque de Smart. C'est par exemple le cas des courbes de Koblitz \footnote{Ou son twist} avec $t = 1$, anomale sur \GF{2} mais pas sur \GF{2^d}. C'est pourquoi on les appelle parfois les courbes anomales binaires (ABC). Insistons bien que l'anomalité dépend du corps des points rationnels considérés. Par exemple une courbe anomale sur \GF{q} ne sera pas anomale sur \GF{q^2}.

Pour les courbes anomales, Smart et al. ont réussi à construire un isomorphisme facilement calculable entre $(E(\GF{q}, +)$ et $(\GF{q}, +)$. On peut ainsi transférer le DL sur $E(\GF{q})$ en un DL sur $(\GF{q}, +)$ qui requiert seulement une inversion.



\subsection{Transfert multiplicatif}
L'attaque de MOV / Frey-Rück, présentée en 1993, permet de transférer le DLP sur $E(\GF{q})$ en un DLP sur $(\GF{q^k}^{*}, \times)$. Sur un tel groupe, il existe des algorithmes, basés sur le calcul d'index, de complexité sous-exponentielle. L'entier $k$, appelé \emph{degré de plongement}, mesure la faisabilité de l'attaque. Pour les courbes admettant un degré de plongement proche de $1$, l'attaque MOV est plus performante que les algorithmes standards. 

Soit $E$ une courbe elliptique sur \GF{q}. Soient $P, R \in E(\GF{q})[l]$. On suppose que $pgcd(l, q) = 1$. On a ainsi : 
\begin{equation}
\langle [m]P, R \rangle = \langle P, R \rangle^{m} \in \mu_l \in \GF{q^k} \text{où $k$ est le degré de plongement.}
\end{equation}
Si on choisit $R$ tel $\langle P, R \rangle \neq 1$, on se ramène au problème du logarithme discret sur $\GF{q^k}^{*}$. 


\begin{align}
\langle P \rangle \subseteq & E(\GF{q}) \\
E[N] \simeq & \left ( \Zn{N} \right )^2 \\
\langle P \rangle \subseteq & E[N] \subseteq E(\GF{q^m})
\end{align}

\begin{itemize}[label=$\bullet$]
    \item Vérifier l'existence de $m$ ? $Q = [m]P \Leftrightarrow [l]Q = \infini{} \text{ et } e_l(P, Q) = 1$
    \item Comment retrouver $m$ ?
    \begin{itemize}[label = --]
        \item Si on arrive à construire $T_1 \in E[l]$\footnote{$T_1$ ne peut pas être un multiple de P, sinon $e_l(P, T_1) = 1$.}, alors $e_l(Q, T_1) = e_l(P, T_1)^m$. Le calcul du logarithme discret dans \GF{q^m} nous donne $k \mod ord(T_1)$.
        \item Construire $T_1$ ? Choisir aléatoirement $T \in E(\GF{q^m})$. Calculer l'ordre $M$ de $T$. Notons $d = pgcd(M, l)$. Posons $T_1 = (M/d)T$. $T_1$ est d'ordre d divisant l.
        \item On réitère ce procédé jusqu'à obtenir l'entier $m$.
    \end{itemize}
    \item $d = 1$ ?
    \item Au final, on transfert un log discret sur $(E(\GF{q}), +)$ en un log discret sur $(\GF{q^m}^*, \times)$. Si $m$ est petit, le calcul d'indice permet de calculer un tel log discret.
\end{itemize}

Menezes, Okamoto et Vanstone ont également montré que le degré de plongement d'une courbe supersingulière est toujours plus petit que $6$; ce type de courbe est donc à proscrire pour une utilisation cryptographique. Nous verrons néanmoins que ces courbes sont tout de même employées dans le contexte des couplages. C'est justement leur petit degré de plongement ainsi que la facilité à les générer qui les rends attrayante pour calculer efficacement les couplages. 

\begin{definition}
Une courbe $E$ sur \GF{q} est dite \emph{supersingulière} lorsqu'elle n'a aucun point de p-torsion. Ou de manière équivalente $a \equiv 0 \pmod p \Leftrightarrow \# E(\GF{q}) \equiv 1 \pmod p$. Si la caractéristique est strictement supérieur à 3, cela revient à dire que la trace de Frobenius est nulle ou que $\# E(\GF{p}) = p + 1$.
\end{definition}

\begin{theoreme}
Le degré de plongement d'une courbe supersingulière est inférieur ou égal à $6$.
\begin{itemize}[label=--]
    \item Si char($\field{}) = 2$, alors $k \leq 4$.
    \item Si char($\field{}) = 3$, alors $k \leq 6$.
    \item Si corps premiers avec $p \leq 5$, alors $k \leq 2$.
\end{itemize}
Les bornes sont atteintes.
\end{theoreme}

\begin{ex}
Soit $p$ un nombre premier de $256$ bits.
Une courbe elliptique sur \GF{p} est censée fournir $128$ bits de sécurité (attaques génériques), équivalent à RSA $3072$.
Si la courbe est supersingulière avec $m = 2$. Alors l'attaque MOV correspond à un log discret sur un corps fini de $512$ bits. Ce qui donne $64$ bits de sécurité (équivalent à RSA $512$).
\end{ex}


\subsection{Attaque GHS}
GHS : Gaudry, Hess, Smart en 2002.
\begin{itemize}
    \item \href{http://citeseerx.ist.psu.edu/viewdoc/download;jsessionid=0FFAE41880B0197BDD30BF31C899B204?doi=10.1.1.25.7356&rep=rep1&type=pdf}{Paper 1}
    \item \href{http://citeseerx.ist.psu.edu/viewdoc/download?doi=10.1.1.38.1805&rep=rep1&type=pdf}{Paper 2}
    \item \href{https://eprint.iacr.org/2001/054}{Paper 3}
\end{itemize}

L'attaque GHS vise à transférer un log discret sur une courbe elliptique définie sur \GF{q^n} vers un log discret sur une courbe hyperelliptique de genre $g$ définie sur \GF{q}. Si $g \geq 4$ (et pas trop grand), alors l'existence des calculs d'indices permettent de résoudre plus efficacement le log discret que les algorithmes génériques


\section{Multiplication complexe}



\section{Rigidité}