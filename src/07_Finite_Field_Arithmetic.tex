\chapter{Arithmétique sur corps fini}
Allez jeter un oeil au Handbook.
\begin{itemize}[label=--]
    \item Multiplication de Karatsuba
    \item Réduction de Montgomery
    \item Multiplication de Montgomery (intéressant seulement pour exponentiation modulaire)
    \item Utilisation de modules spéciaux : nombres de Crandall, nombres de Mersenne généralisé
    \item Inversion binaire dans \GF{p}
    \item Cas de la caractéristique 2 : base polynomial ou base normale.
\end{itemize}

\section{Représentation des entiers}
Représentation d'un entier en base b, bit le plus et le moins significatif. Connaître les bits les plus significatifs d'un entier est équivalent à connaître un intervalle dans lequel se situe cet entier. Plus on connaît de bits et plus l'intervalle sera précis. Un processeur opère sur plusieurs octets par le biais d'un registre. La taille d'un registre est un mot, c'est l'une des principales caractéristiques d'une puce.

Concept de la représentation multi-préçision : permet de manipuler des entiers qui ne peuvent pas être stockés sur un seul mot. Dans ce cas, un tableau de mots est alloué pour stocker l'entier en question. On note $w$ la taille d'un mot. On a $w \in {8, 16, 32, 64}$. Les entiers sont écrits en base $2^w$.

Endianness : détermine l'ordre de représentation. Big-Endian : ordre naturelle soit on commence par le big.

Les ordinateurs sont munis d'opérations très optimisées pour les entiers en simple précision. C'est ce qu'on appelle le \emph{jeux d'instruction} ou instruction CPU. Il dépend du processeur. Sur certaines architectures, les registres flottants sont plus rapides que les registres entiers et doivent donc être utilisé les opérations multi-précision. 

\section{Multiplication d'entiers}
La multiplication de Karatsuba-Ofman permet de décomposer une multiplication en trois multiplications avec des opérandes de tailles réduites. \'A partir d'un certain seuil, ces multiplications sont calculées par la méthode traditionnelle Schoolbook. Le seuil dépend essentiellement du processeur utilisé.

D'autres algorithmes de multiplication existent possédant une complexité asymptotique meilleure que Karatsuba. Cependant ces algorithmes (FFT Transformé de Fourier Rapide, méthode de Tom-Cook) ne sont pas intéressant pour la taille d'entiers utilisés en cryptographie (pas suffisamment grand). 

Complexité quadratique \grandO{n^2}.

\todo{Faire schéma complexité}

\section{Multiplication modulaire}
Deux possibilités sont envisageables : 
\begin{itemize}
    \item On débute par une multiplication dans \Z{} puis on enchaîne avec une réduction. Par exemple on un Karatsuba (ou Schoolbook) suivi d'un Montgomery, Barret, Quisquater ou forme spéciale.
    \item Multiplication intercaler par des réductions. Un algorithme qui combine la multiplication et la réduction.
\end{itemize}

\section{Réduction modulo p}
\subsection{Réduction de Montgomery}
Une implémentation ne doit pas contenir de branchement dépendant de données secrètes. Il faut faire attention à la soustraction finale : toujours l'effectuer même si ça n'est pas nécessaire.

On peut utiliser un nombre premier de la forme $p = 2^\alpha(2^\beta - \gamma) - 1$ où $\alpha, \beta, \gamma$ sont des entiers positifs. Si l'entier $p$ est deux bits plus petit que la taille d'un mot $w$ (soit $w | \alpha + \beta + 2$), alors on peut éviter la soustraction conditionnelle à chaque réduction. 

\subsection{Réduction de Barrett}

\subsection{Forme spéciale}
\begin{itemize}
    \item Nombre de Mersenne. $M_p = 2^p - 1$. Malheureusement leur rareté a conduit à s'intéresser à d'autres nombres premiers. Seules $48$ nombres de Mersenne sont actuellement connu. $M_{127}$ et $M_{521}$ sont premiers, aucun autres n'est premier dans cet intervalle. Pour les tailles ECC, seules $M_{521}$ pourraient être intéressant. Si $a|b$ alors $M_a|M_b$. Ainsi pour que $M_p$ soit premier, une condition nécessaire est que $p$ soit premier. Légende sur $M_{67}$, 3 ans de travail tous les dimanches pour aboutir à une factorisation non triviale. 
    \item Nombre pseudo-Mersenne proposé par Crandall. $M_k = 2^k - c$, $c > 0$.
    \item Nombre de Mersenne Généralisé (GMNs) proposé par Solinas. $p = f(t)$ où $f$ est un polynôme ternaire généralement et $t = 2^k$.
    \item Montgomery-friendly prime : $p = 2\alpha(2\beta - \gamma) - 1$. Lorsque $p$ est deux bits plus courts qu'un multiple de la longueur d'un mot $w$ (ie : $w | \alpha + \beta +2$), alors on peut éviter les soustractions conditionnelles dans chacune des réductions.
\end{itemize}

\section{Inversion modulaire}
L'inversion d'un élément d'un corps fini est une opération particulièrement coûteuse : équivalent à 30 M voire même à plus de 100M dans certains cas. En coordonnée affine, une addition et un doublement de point nécessite chacun une inversion. Afin de réduire ce nombre, on utilise en pratique un système de coordonnées homogènes. Dans ce cas une multiplication scalaire ne demande plus qu'une seule inversion coïncidant à la conversion finale de notre point en affine. 

Cette inversion se calcule généralement via l'égalité donnée par le petit théorème de Fermat : $a a^{p-2} \equiv 1 \pmod p$. Cela se fait en temps constant par le biais d'une chaîne d'addition de $p-2$. Par exemple pour la Curve25519, l'inversion finale prend environ $7\%$ du temps total d'une multiplication scalaire.



\section{Autres}
\subsection{Réduction ou multiplication de Montgomery}
L'idée consiste à remplacer une division coûteuse par des décalages ou autrement dit par des divisions par une puissance de deux. Mais malheureusement cet algorithme de réduction ne peut pas être utilisé tel quel. Dans un premier temps, on change la représentation de nos opérandes et cela peut se faire à l'aide de l'algo de réduction. On calcule ensuite notre multiplication dans \Z puis on réduit. Enfin on revient à la représentation usuelle en utilisant à nouveau l'algo de réduction. 

En plus du changement de représentation, il faut pré-calculer l'inverse de $B$ modulo $N$ et de $N$ modulo $B$. Mais cette étape n'est à faire qu'une seule fois. On comprend alors bien que la réduction de Montgomery trouve tout son intérêt lorsque plusieurs réductions doivent être effectuées avec un petit nombre d'opérande. En particulier la réduction de Montgomery est tout à fait adapté à l'exponentiation modulaire. 

Il existe une technique permettant d'éviter la soustraction conditionnelle de la réduction de Montgomery. \todo{Montgomery exponentiation needs no final substractions, Walter}. Voila ce que dit Walter dans son papier SPA of Unified Code : \og{} When side channels attacks are possible, the final, conditional subtraction should be protected from timing variation by performing the subtraction in all cases if it can occur at all and then selecting the new or old value as appropriate \fg{}. \og{} Now it is clear that constant time modular multiplication is essential for security even when secret keys are always blinded \fg{}. Il faut donc veiller à ce que les opérations dans le corps fini se fasse en temps constant.

\subsection{Inversion Simultanée de Montgomery}
Permet d'inverser $j$ éléments modulo $p$ en faisant un seul Euclide \'Etendu et $3(j-1)$ multiplications au lieu de répéter $j$ fois l'algorithme d'Euclide \'Etendu. L'idée consiste à calculer l'inverse du produit des $a_i$ et en la multipliant par une valeur appropriée on retrouve nos inverses. Pas compliqué.

Notamment utilisé pendant la phase de pré-calcul de la multiplication scalaire.


\section{Arithmétique sur corps binaire}
Les éléments d'un corps non premier sont représentés par un polynôme ou à l'aide d'une base normale. Dans le cas d'un corps binaire \GF{2^n}, c'est un \GF{2}-espace vectoriel de dimension $n$, chaque élément est représenté comme une chaîne de bits de longueur $n$. Le choix de la représentation détermine l'interprétation que l'on fait du motif binaire. Une base normale correspond à l'application itérée du frobenius d'un élément. 

Nous allons nous intéressé à quatre opérations : carré, racine carrée, trace binaire et résolution d'une équation quadratique. 

Dans un corps binaire, via une base normale, un carré correspond à un décalage d'un pas vers la gauche et la racine carrée\footnote{Dans un corps binaire, tout élément admet une unique racine carrée. Tandis que dans un corps non binaire, tout élément non nul admet exactement $2$ ou aucune racine carrée. D'après le frobenius, dans un corps de caractéristique $p$, chaque élément admet exactement une racine $p$-ième.} d'un pas vers la droite. La trace correspond à la somme des frobenius itéré, c'est à dire à la somme d'un point et des points formé par les décalages. Puisque la ième application itérée du frobenius correspond à un décalage de $i$ pas vers la gauche. Au final dans une base normale, la trace correspond à la somme des coordonnées dans chacune de ses composantes.

\'Equation du second degré : changement de variable, admet une solution dans \GF{2^n} ssi la trace sur \GF{2^n} est nul. On peut fournir la formule explicite d'une solution facilement. De plus si $\lambda$ est une solution alors $\lambda + 1$ est l'autre solution.
\begin{itemize}
    \item Représentation polynomiale ou à l'aide d'une base normale.
    \item Calcul d'un carré. Direct pour la base normale. Des astuces sont possibles pour la représentation polynomiale.
    \item Racine carrée. En base normale, correspond à $d-1$ décalages à gauche soit un décalage à droite.
    \item \'Equation quadratique. Changement de variables. Une solution dans \GF{2^d} ssi la trace binaire de l'élément constant est nulle. De plus si $x$ est une solution, alors l'autre vaut $x + 1$.
\end{itemize}