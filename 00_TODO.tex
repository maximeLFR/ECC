\chapter{Questions, Todo List}

\section{Questions}
\begin{itemize}[label=--]
    \item Quelle est la différence entre un isomorphisme sur \field{}, une isogénie sur \field{} et une équivalence birationnelle \field{} ? Les trois notions sont des relations d'équivalence mais de plus en plus faible, c'est à dire que le nombre de classes d'équivalence est de moins en moins grand. Un isomorphisme est une isogénie qui elle-même est une équivalence birationnelle. Les réciproques sont fausses. On les considère uniquement définies sur \GF{p} car sinon elles sont cryptographiquement différentes. 
    \item L'algorithme de Montgomery ladder est-il adapté aux SM à bases multiples ? Si non pourquoi.
    \item \'A quelle condition une courbe d'Edwards peut-elle être transformée en une tordue d'Edwards avec $a=-1$ ? Même question mais en partant d'une tordue d'Edwards avec $a\neq-1$.
    \item Discriminant du corps des endomorphismes. S'il est faible, qu'est-ce que cela signifie : qu'il possède un endomorphisme facilement calculable ?
    \item \'Ecriture générique d'une isogénie (cf. Benjamin Smith).
\end{itemize}


\section{Remarques}



\section{Si j'ai le temps}
\begin{itemize}[label=--]
    \item \'Etude du théorème de Rieman-Roch.
    \item Discriminant d'un polynôme.
    \item Dérivées partielles, implicites.
\end{itemize}