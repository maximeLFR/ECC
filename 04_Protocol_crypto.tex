\chapter{Protocoles sur EC}
\todo{\'Ecrire les algorithmes en Latex.}

\noindent On se donne les paramètres suivants : 
\begin{itemize}[label=$\bullet$]
    \item Soit $E$, une courbe elliptique définie sur un corps fini \GF{q}.
    \item Soit $P$, un \GF{q}-point rationnel d'ordre $l$, où $l$ est le plus grand facteur premier du cardinal de la courbe. $P$ est appelé le \emph{point de base} et le sous-groupe principal est le sous-groupe engendré par $P$.
\end{itemize}

Le \emph{cofacteur} $h$ est l'ordre du groupe $\#E(\GF{p})$ divisé par son plus grand facteur premier $l$. On a $\#E(\GF{p}) = hl$. Pour des raisons d'efficacité, on choisit $h$ le plus petit possible. Par exemple NIST préconise des courbes elliptiques avec un cofacteur égal à un.

Une \emph{clé privée} est entier $k$ choisi aléatoirement dans $\{1, \ldots, l-1\}$. La \emph{clé publique} correspondante est le point $P_k = [k]P$.

\section{Protocoles d'échange de clé}
\todo{Résumé : key establishment. Ou faire ce résumé dans crypto et mettre un lien.}
\subsection{ECDH : Ellptic Curve Diffie Hellman}
ECDH est un protocole d'échange de clé publié en 1976 par Diffie et Hellman dans leur célèbre article \og New directions in Cryptography \fg{}. Cependant Ellis, Cocks et Williamson qui travaillaient pour les services de renseignements britanniques (GCHQ) avaient déjà découvert ce procédé quelques années plus tôt. De plus il faut savoir que malgré le nom trompeur du protocole, que Merkle \footnote{Il est aussi à l'origine du procédé de construction de Merkle-Damgard pour les fonctions de hachages.} a aussi participé à cette découverte.
\begin{algorithm}
    \caption{ECDH}
    \label{ECDH}
    \begin{algorithmic}[1]
    
        \REQUIRE
        \ENSURE 
        \STATE $(i, j)$ \hspace{1.04cm} $\leftarrow $  
    \end{algorithmic}
\end{algorithm}

\begin{itemize}[label=$\bullet$]
    \item Rappelons que l'objectif d'un DH est le partage d'un secret (une chaîne de bits quelconques) entre Alice et Bob. Ce secret n'est jamais utilisé tel quel comme clé privée d'un algorithme symétrique. La raison est que la clé privée d'un algo symétrique doit être de longueur fixé et uniformément distribué (d'entropie maximale). Le secret est utilisé comme l'entrée d'une fonction de dérivation de clé KDF \footnote{Key Derivation Function.} qui construit un ensemble de clé appelé \emph{clé de session}. On peut ainsi changer de clé symétrique sans refaire un ECDH.
    
    \item ECDH, bien plus rapide que DH sur un corps fini, se décline en deux versions. 
        \begin{enumerate}
            \item ECDHE : ECDH éphémère. Protocole habituellement présenté. C'est un accord de clé : les deux parties influencent la valeur du secret partagée. De plus il possède la propriété de cryptographie persistante \footnote{Forward Secrecy} : la compromission de la clé privée statique de long terme ne compromet pas la confidentialité des communications passées. En effet cette clé n'est utilisée que pour signée et n'intervient pas dans la construction du secret. 
            \item ECDH statique ou à un pas. C'est un transport de clé : seul un des participant définit le secret. L'avantage est que les deux parties n'ont pas besoin d'être connecté. Cette version assure uniquement une authenticité unilatérale. Cette caractéristique peut s'avérer intéressante. Par exemple pour le protocole SSL \ldots A revoir.
        \end{enumerate}
        
    \item Protocole vulnérable à une attaque active de type man-in-the-middle. Comment s'assurer que le message reçu (clé publique éphémère) provient bien d'Alice ? Pour éviter cette attaque, il est nécessaire que les messages échangés entre Alice et Bob soient authentifiés. Pour cela, Alice doit signer la clé publique éphémère qu'elle envoie avec ECDSA et sa clé publique de long terme. Elle envoie ainsi : $(aG, (r, s))$. On parle dans ce cas de ECDHE-ECDSA.
    
    \item Cela n'est pas toujours suffisant. Il faut aussi s'assurer de l'authenticité de la clé publique. La clé publique me permettant de vérifier la signature est-elle bien celle d'Alice ? Pour cela on a recourt à la notion de certificat (permet de lier l'identité d'une entité à sa clé publique) et d'une autorité de confiance. On transfert le problème d'authenticité de la clé publique d'Alice à l'authenticité de la clé publique d'une autorité de confiance appelée \emph{clé racine}.
    
    \item Lorsque l'on reçoit une clé publique (un point), il faut s'assurer que c'est un point de la courbe et qu'il est d'ordre $l$ ie : qu'il appartient au sous-groupe principal (Procédure de validation de clé). Un attaquant peut envoyer un point de petit ordre et ainsi calculer le log discret de ce point. Il récupère la valeur de la clé secrète modulo l'ordre de son point. Il peut réitérer ce procédé puis utiliser le CRT. Un point qui n'appartient pas à la courbe ne doit jamais être accepté. Si le point appartient à la courbe mais est d'ordre $h$, l'attaquant ne gagnera pas grand chose, on peut utiliser le ECDH-cofacteur. 
    
    \item En pratique, on utilise généralement des techniques de compression de points. Mais on peut faire encore mieux : courbes de Montgomery.
\end{itemize}

\subsection{ECMQV : Menezes-Qu-Vanstone}
MQV est un protocole d'échange de clé authentifié basé sur le schéma de Diffie-Hellman. Il a été initialement proposé par Menezes, Qu et Vanstone en 1995 et modifié par Law et Solinas en 1998. Schéma breveté par Certicom.

\section{Schémas de chiffrement}
\subsection{EC-ElGamal : Elliptic Curve ElGamal}
Elgamal est un Diffie-Hellman déguisé. L'algorithme de chiffrement consiste à générer une bi-clé éphémère. Le message $m$ à chiffrer correspond soit à un point, soit à un scalaire. Pour le chiffrer, dans le premier cas, on l'additionne avec le secret définit par un DH \footnote{Pour cela, on récupère la clé publique du destinataire.}, dans le second on somme le clair avec l'abscisse du secret. Il reste à envoyer ce point (ou scalaire) ainsi que notre clé publique éphémère.

Le chiffrement est donc non déterministe.

\subsection{Cryptosystème de Cramer et Shoup}
\subsection{ECIES : Elliptic Curve Integrated Encryption Scheme}


\section{Schémas de signature}
\subsection{ECDSA : Elliptic Curve Digital Signature Algorithm}
ECDSA est un schéma de signature proposé par Vanstone en 1992 en réponse à une requête du NIST. Il a été standardisé par plusieurs organismes : ISO en 98, ANSI en 99 et IEEE, NIST en 2000.

Initialement une preuve de sécurité a été trouvée sur des variantes de (EC)DSA. Cependant cette preuve ne s'applique pas (et pas encore actuellement) à la version originale. Durant un temps, il a donc été envisagé d'ajouter quelques modifications dans les standards. Mais aujourd'hui de nouvelles preuves ont été proposées concernant ECDSA qui d'ailleurs ne s'appliquent ni à DSA ni à ces variantes.



\begin{itemize}[label=$\bullet$]
    \item La signature est non déterministe. Un même message n'est pas signé de la même façon. La signature nécessite l'inversion du nonce, des méthodes de types pré-calcul ont été proposées pour accélérer ce process.
    \item La vérification nécessite une multi-exponentiation n'impliquant aucune donnée sensible. Il n'est donc pas nécessaire de se soucier des attaques par canaux auxiliaires, pas besoin d'implémentation en temps constant. 
    \item Il est primordial que la clé éphémère générée pour signer un message doit être parfaitement aléatoire. On l'appelle un nonce (usage unique). Si la clé éphémère est utilisée plusieurs fois pour signer un même message. Alors on peut retrouver la clé. La playstation 3 de Sony a été cassée à cause d'une mauvaise implémentation : le nonce était constant. Il était ainsi possible de retrouver la clé secrète aisément.
    \item Partial key exposure. Si des bits de la clé éphémère fuient alors une attaque sur les réseaux proposé par Howgrave-Graham et Smart est possible. Elle permet de retrouver la clé secrète à l'aide de quelques signatures. \todo{citer}. Cette méthode a été généralisée par Nguyen et Shparlinski. Il faut que remarquer que cette fuite de bits pourrait être envisager via une SCA ou à cause d'un mauvais générateur pseudo aléatoire.
    \item Il existe aussi une attaque de Bleichenbacher montrant que la présence d'un biais dans la génération du nonce peut être utilisée pour retrouver la clé secrète de long terme.
    \item ECDSA vs RSA : la vérification des signatures via RSA est plus efficace \footnote{e = 3 ou 65537 = $2^{16} + 1$ possède un faible poids de Hamming.} que ECDSAdd
    \item $r$ et $s$ doivent être non nul.
    \item Il existe des variantes. ECGDSA  externalise le calcul de l'inverse à la procédure de génération de clé. Cela permet de rendre plus efficace la phase de signature.
\end{itemize}


\subsection{Schéma de signature de Schnorr}

\section{Générateur pseudo-aléatoire}
\todo{Décrire le PNRG EC-Dual-DRBG et sa potentielle trappe secrète introduite par la NSA.} Cf mes favoris.

\section{Sécurité}
La sécurité de ECC dans le modèle en boîte noire repose sur la difficulté de l'un des problèmes suivants :
\begin{itemize}[label=$\bullet$]
    \item Problème du Logarithme Discret sur Courbes Elliptiques (ECDLP) : calcul de $k$ étant donné $P$ et $Q = [k]P$.
    \item Diffie-Hellman Calculatoire sur Courbes Elliptiques (ECCDH) : calcul de $[k_1k_2]P$ étant donné $P$, $Q_1 = [k_1]P$ et $Q_2 = [k_2]P$.
    \item Diffie-Hellman Décisionnel sur Courbes Elliptiques (ECDDH) : étant donné $P$, $Q_1 = [k_1]P$ et $Q_2 = [k_2]P$ et $Q_3 = [k_3]P$, évaluer si $Q_3 = [k_1k_2]P$.
\end{itemize}

Ces problèmes sont énoncés de manière décroissante par rapport à leurs difficultés. Si je sais résoudre ECDLP, alors il en est de même de ECCDH et ECDDH. Mais on ne sait rien actuellement sur les réciproques.