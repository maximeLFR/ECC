\chapter{Arithmétique sur EC}
La multiplication scalaire correspond à une séquence d'addition et de doublement. Le choix de l'algorithme d'exponentiation détermine cette séquence. Dans ce chapitre, nous nous situons à un niveau inférieur, nous nous intéressons aux formules d'addition et de doublement de points ainsi qu'aux améliorations algorithmiques possibles. 

Le coût de ces opérations se mesure en terme de nombres d'opérations élémentaires dans le corps de base. Ceci permet de nous affranchir du matériel utilisé, c'est à dire de comparer l'efficience d'algorithmes quelque soit le matériel.

\begin{table}[h]
\centering
    \begin{tabular}{|l|l|}
     Nom & Description de l'opération \\
     \hline \hline
     ADD & Coût d'une addition.\\
     mADD & Mixed addition. Lorsque $Z_2 = 1$.\\
     mmADD & Lorsque $Z_1 = Z_2 = 1$.\\
     CoZ-ADD & Lorsque $Z_1 = Z_2$. Due à Méloni en 2007. \\
     reADD & Coût d'une ré-addition. \\
      & Une addition dans laquelle une des opérandes a déja été additionnée préalablement. \\
     \hline
     DBL & Coût d'un doublement.\\
     mDBL & Coût d'un doublement lorsque $Z_1 = 1.$\\
     \hline
     SCALE & Coût de la conversion en affine.\\
     & Nécessite toujours au moins une conversion.\\
     \hline
    \end{tabular}
\caption{Différentes opérations sur une EC}
\label{tab:operation}
\end{table}

Chacun de ces coûts est exprimé en fonction du nombre d'opération dans le corps de base. On ne tient compte que des inversions (I), des multiplications (M) et des carrés (S); les autres opérations étant supposées négligeables. Il est d'usage de considérer que $S = 0.8M$\footnote{Le ratio $S/M$ est quasi-indépendant du corps et de l'implémentation. Tandis que le ratio $I/M$ en dépend.} pour un corps premier et $S = 0.2 M$ ou $S = 0M$ pour un corps binaire\footnote{Si les éléments sont représentés à l'aide d'une base normale, alors un carré correspond à un décalage d'un pas vers la gauche.}, tandis que le ratio $I/M$ lui dépend du matériel utilisé. On peut néanmoins affirmer qu'une inversion coûte sensiblement plus cher qu'une multiplication. C'est pourquoi différents systèmes de coordonnées ont été introduit, afin de ne plus avoir à faire d'inversion. Pour une multiplication scalaire, (au moins) une inversion restera tout de même nécessaire afin de convertir le point en coordonnée affine.

Au delà des opérations présentes dans \ref{tab:operation}, des améliorations calculatoires sont possibles pour d'autres types de calculs. Par exemple pour le triplement, doublement suivi d'une addition effectué en une seule fois, plusieurs doublement consécutif \ldots 

Le site \og \href{http://www.hyperelliptic.org/EFD/}{Explicit-Formulas Database} \fg{}, EFD, contient une collection de formules explicites pour ces opérations. Référence en la matière, on y trouve les formules les plus efficaces connues, ainsi que des accélérations découvertes par les auteurs publiées uniquement sur le site EFD.


\section{Systèmes de coordonnées}
\subsection{Coordonnées projectives}
Aussi appelée coordonnées homogènes, $(X : Y : Z) \sim (\lambda X : \lambda Y : \lambda Z)$. Les points affines sont les points tel que $Z = 1$, ainsi $(x, y) = (X/Z, Y/Z)$.

L'opposé d'un point $(X : Y : Z)$ est donné par $(X : -Y : Z)$ sur un corps premier et par $(X : X + Y : Z)$ sur un corps binaire. Le point à l'infini est $\infini = (0 : 1 : 0)$.

Les racines d'un polynômes à coefficient dans un espace quotient ne sont pas toujours bien défini. C'est à dire que si un polynôme s'annule sur un point $P$, il doit aussi s'annuler pour les élément appartenant à cette classe d'équivalence. Autrement dit la nullité d'un polynôme doit être indépendant du représentant choisi. Si $F(X : Y : Z) = 0$, alors $F(\lambda X : \lambda Y : \lambda Z) = 0$ pour tout $\lambda \in \GF{p}$. Pour cela on considère le polynôme homogène associé.


\subsection{Coordonnées jacobiennes}
En coordonnées Jacobiennes, les points vérifient $(X : Y : Z) \sim (\lambda^2 X : \lambda^3 Y : \lambda Z)$ pour tout $\lambda$ appartenant à \GF{p}. On a $(x, y) = (X/Z^2, Y/Z^3)$.
De la même manière qu'avec les coordonnées projectives, le point à l'infini est donné lorsque $Z = 0$ et on retrouve les coordonnées affines lorsque $Z = 1$.



De manière générale, on a $(X : Y : Z) \sim (\lambda^a X : \lambda^b Y : \lambda^c Z)$. Les paramètres $a, b, c$ déterminent le degré de $X$, $Y$ et $Z$. On homogénéise un polynôme en veillant à ce que chaque monôme ait un degré égal. Pour cela on ajoute des termes en $Z$. On peut ainsi obtenir un polynôme homogène de degré égal au $ppcm(a, b, c)$ où un de ses multiples.

\begin{table}[h]
\centering
\begin{tabular}{c|c|c|c}
Syst. Coordonnées & Add & Dbl & Dbl-3 \\
\hline \hline
Projective & $12M + 2S \simeq 13,6M$ & $5M + 6S \simeq 9,8M$ & $7M + 3S \simeq 9,4M$\\
Jacobienne & $11M + 5S \simeq 15M$   & $1M + 8S \simeq 7,4M$ & $3M + 5S \simeq 7M$\\
\hline
\end{tabular}
\caption{Coût de l'addition de points d'une courbe elliptique}
\label{tab:cost}
\end{table}

On observe que le coût d'un doublement est nettement inférieur à une addition (plus du double pour les coordonnées Jacobiennes).

\subsection{Autres systèmes de coordonnées}
Pour les corps premiers, il existe aussi les coordonnées jacobiennes chudnovsky et les jacobiennes modifiées.


\subsection{Remarques}
\begin{itemize}[label=$\bullet$]
    \item Mixed coordinate. L'idée consiste à utiliser plusieurs systèmes de coordonnées. Par exemple, on peut faire un doublement en coordonnée affine et représenter le résultat en coordonnée jacobienne.
    \item Précalcul. Pour calculer une multiplication scalaire, on a souvent recourt à un sliding window signé (sliding window combiné à un NAF). Dans ce cas, il est nécessaire de précalculer les points $[k]P$ avec k impaire, jusqu'à un certain niveau. De plus parfois ces précalculs ne peuvent pas être fait offline car le point de base n'est pas fixé\footnote{C'est le cas pour ECDH ou la vérification d'ECDSA par exemple.}. L'efficacité de l'étape de précalcul est donc importante. Pour des raisons d'efficacité, les points précalculés sont représentés en affine ou en jacobienne chudnovsky. L'article de 2007 \href{}{}
    \item $a_4 = -3$. Choisir une courbe elliptique vérifiant $a = -3$ n'est pas restrictif. Une courbe $E_{a, b}$ est \GF{p}-isomorphe à une courbe $E_{-3, b^{'}}$ ssi $-3/a$ a une racine quatrième dans \GF{p}. Ce qui est vérifié, pour une valeur de $a$ sur quatre lorsque $q \equiv 1 \pmod 4$ et une sur deux lorsque $q \equiv 3 \pmod 4$.

On peut aussi passer par la notion d'isogénie. Deux courbes sur $\GF{}$ sont isogènes ssi elles ont le même cardinal. Ainsi le DLP sur une courbe isogène est aussi difficile que sur la courbe originale. Puisque le problème du logarithme discret sur une courbe elliptique dépend uniquement du cardinal de la courbe tout comme le degré de plongement. De plus pour toute isogénie $\phi : E \to E^{'}$ de degré $m$, il existe une unique isogénie $\hat{\phi}$, appelé isogénie duale, vérifiant : $\hat{\phi} \circ \phi = [m]$ et $\phi \circ \hat{\phi} = [m]^{'}$. L'idée consiste ainsi à déterminer une  isogénie $\phi$ de degré petit $m$ et à valeur dans une courbe elliptique avec $a_4 = -3$. L'entier $m$ est très probablement inversible modulo l'ordre du point de base, on note ainsi $k_m \equiv k/m \pmod{ord(P)}$. Au final, on peut décomposer la multiplication scalaire par $k$ via : $[k] = [k_m m] = [k_m] \circ [m] = \hat{\phi} \circ [k_m] \circ \phi$. La multiplication par $k_m$ s'effectue quant à elle avec le paramètre $a_4$ égal à $-3$. 
    \item Compression. Dans les différents protocoles sur courbes elliptiques, il est souvent nécessaire de transmettre des points à travers un canal. Ceci a bien entendu un coût en terme de bande passante. Pour limiter ce coût, on peut se limiter à transmettre l'abscisse et un bit de $y$. Ces données sont suffisantes afin de recouvrer le point correspondant. 
\end{itemize}



\todo{Faire un tableau résumant le nombre d'opération de base nécessaire pour l'addition et le doublement pour chaque systèmes de coordonnées et ceux pour chaque formes de courbes. Faire un tableau comparatif des différents modèles de courbes.}