%\newcommand\nomCommande[nbArgument][ValeurParDef]{CodeLatex}
% #i est le nom de la ieme variable. i <= 9. Valeur par défaut
% appel de la commande : \nomCommande{arg1}{arg2}
% package xargs permet de s'affranchir de la limite du nombre de paramètre

%%%%%%%%%%%%%%%%%%%
% 1. Theorem
%%%%%%%%%%%%%%%%%%%
\newtheorem{definition}{Définition}
\newtheorem{theoreme}{Théorème}
\newtheorem{preuve}{Preuve}
\newtheorem{propriete}{Propriété}
\newtheorem{corollaire}{Corollaire}
\newtheorem{ex}{Exemple}
\newtheorem{remarque}{Remarques}
% utilisation via : \begin{definition} ... \end{definition}
%%%%%%%%%%%%%%%%%%%%%


%%%%%%%%%%%%%%%%%%%
% 2. Commandes
%%%%%%%%%%%%%%%%%%%
% Commandes Latex pour les mathématiques
\newcommand\grandO[1]{\ensuremath{O\mathopen{}\left(#1\right)}}

\newcommand{\fonction}[5]{\begin{array}{l|ccl}
#1: & #2 & \longrightarrow & #3 \\
    & #4 & \longmapsto & #5 \end{array}}
    
\newcommand{\intervalle}[4]{\ensuremath{\mathopen{#1}#2\mathclose{}\mathpunct{};#3\mathclose{#4}}}
\newcommand{\intervalleff}[2]{\intervalle{[}{#1}{#2}{]}}
\newcommand{\intervalleof}[2]{\intervalle{]}{#1}{#2}{]}}
\newcommand{\intervallefo}[2]{\intervalle{[}{#1}{#2}{[}}
\newcommand{\intervalleoo}[2]{\intervalle{]}{#1}{#2}{[}}
\newcommand{\intervalleentier}[2]{\intervalle\llbracket{#1}{#2}\rrbracket}

\newcommand{\infini}{\ensuremath{O}}
\newcommand{\ensnombre}[1]{\ensuremath{\mathbb{#1}}}
\newcommand{\N}{\ensnombre{N}}
\newcommand{\Z}{\ensnombre{Z}}
\newcommand{\Q}{\ensnombre{Q}}
\newcommand{\R}{\ensnombre{R}}
\newcommand{\C}{\ensnombre{C}}
\newcommand{\field}[1][K]{\ensnombre{#1}}
\newcommand{\GF}[1]{\ensuremath{\ensnombre{F}_{#1}}}
\newcommand{\Zn}[1]{\ensuremath{\Z/#1\Z}}
%\frac{\Z}{#1\Z}}}
\newcommand{\closure}[1][\ensnombre{K}]{\ensuremath{\overline{#1}}}
\newcommand{\CE}[1][\closure]{\ensuremath{E(#1)}}

\newcommand{\abs}[1]{\left\lvert#1\right\rvert}
\newcommand{\norme}[1]{\left\lVert#1\right\rVert}

\newcommand{\enstq}[2]{\left\{#1\mathrel{}\middle|\mathrel{}#2\right\}}

\newcommand{\prodscal}[2]{\left\langle#1,#2\right\rangle}

\DeclareMathOperator*{\argmin}{arg\,min}

\newcommand{\todo}[1]{\textcolor{red}{#1}}